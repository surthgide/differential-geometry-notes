\exer{Prodotto diretto di gruppi di Lie}
{exer3-1}
{
Dimostrare che il prodotto diretto di due gruppi di Lie è un gruppo di Lie.
}
{
\textit{soluzione}
}

%=======================================================================================

\exer{Topologie sul toro}
{exer3-2}
{
Siano la proiezione

\map{\pi}
	{\R^{2}}{\T^{2} = \S^{1} \times \S^{1}}
	{(t,s)}{(e^{2 \pi i t},e^{2 \pi i s})}

l'insieme

\begin{equation}
	L = \{ (t,\alpha t) \in \R^{2} \mid \alpha \in \R \setminus \Q \}
\end{equation}

e la restrizione di $ \pi $ a $ L $, i.e.

\begin{equation}
	f = \eval{\pi}_{L} : L \to \S^{1} \times \S^{1}
\end{equation}

Siano

\begin{itemize}
	\item $ \tau_{f} $ la topologia indotta da $ f $ su $ H = \pi(L) $;
	
	\item $ \tau_{s} $ la topologia indotta dall'inclusione $ H \subset \S^{1} \times \S^{1} $
\end{itemize}

Dimostrare che $ \tau_{s} \subset \tau_{f} $.
}
{
\textit{soluzione}
}

%=======================================================================================

\exer{Esponenziale di matrice}
{exer3-3}
{
Sia la matrice

\begin{equation}
	X = \bmqty{ %
				0 & 1 \\
				1 & 0 %
				}
\end{equation}

Dimostrare che

\begin{equation}
	e^{X} = \bmqty{ %
					\cosh(1) & \sinh(1) \\ \\
					\sinh(1) & \cosh(1) %
					}
\end{equation}
}
{
\textit{soluzione}
}

%=======================================================================================

\exer{Esponenziale di somma di matrici}
{exer3-4}
{
Trovare due matrici $ A $ e $ B $ tali che

\begin{equation}
	e^{A+B} \neq e^{A} e^{B}
\end{equation}
}
{
\textit{soluzione}
}

%=======================================================================================

\exer{Matrici unitarie come insieme compatto}
{exer3-5}
{
Dimostrare che il gruppo unitario $ U(n) $ è compatto per ogni $ n \geqslant 1 $.
}
{
\textit{soluzione}
}

%=======================================================================================

\exer{Proprietà componente connessa $ G_{0} $ di un gruppo di Lie}
{exer3-6}
{
Siano $ G $ un gruppo di Lie e $ G_{0} $ la componente connessa di $ G $ che contiene $ e $ (elemento neutro di $ G $). Se $ \mu $ e $ i $ denotano rispettivamente la moltiplicazione e l'inversione in $ G $, provare che:

\begin{enumerate}
	\item $ \mu(\{g\} \times G_{0}) \subset G_{0} $ per qualsiasi $ g \in G_{0} $;
	
	\item $ i(G_{0}) \subset G_{0} $;
	
	\item $ G_{0} $ è un sottoinsieme aperto di $ G $;
	
	\item $ G_{0} $ è un sottogruppo di Lie di $ G $.
\end{enumerate}
}
{
\textit{soluzione}
}
}

%=======================================================================================

\exer{Diffeomorfismo $ SO(2) \simeq \S^{1} $}
{BONUS3-1}
{
Verificare che $ SO(2) $ sia diffeomorfo a $ \S^{1} $.
}
{
\textit{soluzione}
}

%=======================================================================================

\exer{Diffeomorfismo $ SU(2) \simeq \S^{3} $}
{BONUS3-2}
{
Verificare che $ SU(2) $ sia diffeomorfo a $ \S^{3} $.
}
{
\textit{soluzione}
}

%=======================================================================================

\exer{Differenziale moltiplicazione gruppo di Lie}
{exer3-7}
{
Sia $ G $ un gruppo di Lie con moltiplicazione $ \mu : G \times G \to G $. Dimostrare che

\begin{equation}
	\mu_{*(a,b)}(X_{a},Y_{b}) = (R_{b})_{*a}(X_{a}) + (L_{a})_{*b}(Y_{b}) %
	\qcomma \forall (a,b) \in G \times G, \, \forall X_{a} \in T_{a}(G), \, \forall Y_{b} \in T_{b}(G)
\end{equation}

dove $ L_{a} $ (risp. $ R_{b} $) denota la traslazione a sinistra (risp. a destra) associata ad $ a $ (risp. $ b $).
}
{
\textit{soluzione}
}

%=======================================================================================

\exer{Differenziale inversione gruppo di Lie}
{exer3-8}
{
Sia $ G $ un gruppo di Lie con inversione $ i : G \to G $. Dimostrare che

\begin{equation}
	i_{*a}(Y_{a}) = -(R_{a^{-1}})_{*e}(L_{a^{-1}})_{*a} (Y_{a}) %
	\qcomma \forall a \in G, \, \forall Y_{a} \in T_{a}(G)
\end{equation}
}
{
\textit{soluzione}
}

%=======================================================================================

\exer{Commutatore matriciale per algebre di Lie}
{exer3-9}
{
Verificare che il commutatore tra matrici

\begin{equation}
	[A,B] = AB - BA
\end{equation}

definisce un'algebra di Lie sullo spazio tangente all'identità dei gruppi: $ O(n) $, $ SO(n) $, $ U(n) $, $ SU(n) $, $ SL_{n}(\R) $, $ SL_{n}(\C) $.
}
{
\textit{soluzione}
}

%=======================================================================================

\exer{Esponenziale per gruppi di Lie}
{exer3-10}
{
Verificare che l'esponenziale di una matrice definisce un'applicazione

\map{e}
	{T_{I}(G)}{G}
	{A}{e^{A}}

per i gruppi di Lie: $ GL_{n}(\R) $, $ GL_{n}(\C) $, $ SL_{n}(\R) $, $ SL_{n}(\C) $, $ O(n) $, $ SO(n) $, $ U(n) $, $ SU(n) $
}
{
\textit{soluzione}
}

%=======================================================================================

\exer{Algebra di Lie di prodotto diretto di gruppi di Lie}
{exer3-11}
{
Sia $ G = G_{1} \times \cdots \times G_{s} $ il prodotto diretto di gruppi di Lie. Dimostrare che l'algebra di Lie di $ G $ è isomorfa alla somma diretta delle algebre di Lie dei $ G_{i} $ con $ i=1,\dots,s $.
}
{
\textit{soluzione}
}

%=======================================================================================

\exer{Parallelizzabilità dei gruppi di Lie}
{exer3-12}
{
Dimostrare che ogni gruppo di Lie è parallelizzabile.
}
{
\textit{soluzione}
}
