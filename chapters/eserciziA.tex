\exer{Funzione $ C^{k}(\R) $ ma non $ C^{k+1}(\R) $}
{exer1-1}
{%
Per ogni numero naturale $ k \in \N $ costruire una funzione $ C^{k}(\R) $ ma non $ C^{k+1}(\R) $. %
}
{
\textit{soluzione}
}

%=======================================================================================

\exer{Funzione liscia ma non reale analitica}
{exer1-2}
{
Dimostrare che la funzione

\map{f}
	{\R}{\R}
	{x}{%
		\begin{cases}
			e^{\sfrac{-1}{x^{2}}} & x \neq 0 \\
			0 & x = 0
		\end{cases}
	}

risulta essere liscia ma non reale analitica.
}
{
\textit{soluzione}
}

%=======================================================================================

\exer{Intervalli diffeomorfi a $ \R $}
{exer1-3}
{
Siano $ a,b,c,d \in \R $ tale che $ a<b $. Dimostrare che i seguenti intervalli sono tutti diffeomorfi tra loro e diffeomorfi a $ \R $:

\begin{equation}
	\begin{cases}
		(a,b) \\
		(c,+\infty) \\
		(-\infty,d)
	\end{cases}
\end{equation}
}
{
\textit{soluzione}
}

%=======================================================================================

\exer{Diffeomorfismo tra $ B_{r}(c) $ e $ \R^{n} $}
{exer1-4}
{
Dimostrare che l'applicazione

\map{h}
	{B_{1}(0)}{\R^{n}}
	{x}{\left( \dfrac{x^{1}}{\sqrt{1 - \norm{x}^{2}}}, \cdots, \dfrac{x^{n}}{\sqrt{1 - \norm{x}^{2}}} \right)}

definisce un diffeomorfismo tra la palla aperta unitaria centrata nell'origine di $ \R^{n} $ ed $ \R^{n} $. Dedurre che la palla aperta di centro $ c \in \R^{n} $ e raggio $ r > 0 $ in $ \R^{n} $ è diffeomorfa a $ \R^{n} $.
}
{
\textit{soluzione}
}

%=======================================================================================

\exer{Teorema di Taylor con resto per funzione a due variabili}
{exer1-5}
{
Sia $ f \in C^{\infty}(\R^{2}) $. Usando il teorema di Taylor con resto, dimostrare che esistono $ g_{11},g_{12},g_{22} \in C^{\infty}(\R^{2}) $ tali che

\begin{equation}
	f(x,y) = f(0,0) + x \, \dfrac{\partial f}{\partial x} (0,0) + y \, \dfrac{\partial f}{\partial y} (0,0) + x^{2} \, g_{11}(x,y) + x y \, g_{12}(x,y) + y^{2} \, g_{22}(x,y)
\end{equation}
}
{
\textit{soluzione}
}

%=======================================================================================

\exer{Funzione liscia tramite incollamento}
{exer1-6}
{
Sia $ f \in C^{\infty} (\R^{2}) $ tale che

\begin{equation}
	f(0,0) = \pdv{f}{x} \, (0,0) = \pdv{f}{y} \, (0,0) = 0
\end{equation}

Sia l'applicazione

\map{g}
	{\R^{2}}{\R}
	{(t,u)}{%
			\begin{cases}
				\dfrac{f(t,tu)}{t} & t \neq 0 \\ \\
				0 & t = 0
			\end{cases}
			}

Dimostrare che $ g \in C^{\infty}(\R^{2}) $.
}
{
\textit{soluzione}
}
}

%=======================================================================================

\exer{$ C_{p}^{\infty}(\R^{n}) $ come algebra commutativa e unitaria}
{exer1-7}
{
Dimostrare che l'insieme $ C_{p}^{\infty}(\R^{n}) $ dei germi delle funzioni lisce intorno a $ p \in \R^{n} $ con le operazioni di somma e di prodotto definite negli appunti è un'algebra commutativa e unitaria.
}
{
\textit{soluzione}
}

%=======================================================================================

\exer{$ \der_{p}(C_{p}^{\infty}(\R^{n})) $ come spazio vettoriale su $ \R $}
{exer1-8}
{
Dimostrare che l'insieme $ \der_{p}(C_{p}^{\infty}(\R^{n})) $ delle derivazioni puntuali con le operazioni definite negli appunti è uno spazio vettoriale su $ \R $.
}
{
\textit{soluzione}
}

%=======================================================================================

\exer{$ \chi(U) $ come spazio vettoriale su $ \R $ e $ C^{\infty} $-modulo}
{exer1-9}
{
Dimostrare che l'insieme dei campi di vettori lisci $ \chi(U) $ su un aperto $ U \subset \R^{n} $ con le operazioni definite negli appunti è uno spazio vettoriale su $ \R $ e un $ C^{\infty} $-modulo.
}
{
\textit{soluzione}
}

%=======================================================================================

\exer{$ \der(A) $ come spazio vettoriale su $ \K $}
{exer1-10}
{
Sia $ A $ un'algebra su un campo $ \K $. Dimostrare che le operazioni

\begin{equation}
	\begin{cases}
		(D_{1}+D_{2})(a) = D_{1}(a) + D_{2}(a) \\
		(\lambda D)(a) = \lambda D(a)
	\end{cases} %
	\qquad \forall \lambda \in \K, \, \forall D_{1},D_{2},D \in \der(A)
\end{equation}

dotano $ \der(A) $ della struttura di spazio vettoriale su $ \K $.
}
{
\textit{soluzione}
}

%=======================================================================================

\exer{Commutatore come derivazione}
{exer1-11}
{
Siano $ D_{1} $ e $ D_{2} $ due derivazioni di un'algebra $ A $ su un campo $ \K $, i.e. $ D_{1},D_{2} \in \der(A) $. Mostrare che $ D_{1} \circ D_{2} $ non è necessariamente una derivazione di $ A $ mentre

\begin{equation}
	D_{1} \circ D_{2} - D_{2} \circ D_{1} \in \der(A)
\end{equation}
}
{
\textit{soluzione}
}
